\documentclass[10pt,dvipdfmx]{jarticle}
\usepackage{ascmac}
\usepackage{amsmath}
\usepackage{listings}
\usepackage{graphicx}

\title{信号処理特論 第13回課題}
\author{視覚認知システム研究室\\学籍番号:2233730006 千畑 颯也}
\date{\today}

\begin{document}
  \maketitle
  \begin{itembox}[l]{課題3}
    平均0,分散1のガウス乱数$x(n)(n=0,1, \cdots ,999)$を入力信号として,出力を
    \[y(n) = \sum^{N-1}_{m=0}g(n-m)x(m) + \epsilon (n) = \sum^{N-1}_{m=0}g(m)x(n-m) + \epsilon (n) \]
    作る.ただし,$N=5$であり,$(g(0),g(1),g(2),g(3),g(4))=(1,-0.9,0.8,-0.7,0.6)$,$x(m)=0(m<0)$とする.また,$\epsilon (n)$は$x(n)$とは独立なガウス乱数を$1/20$倍したものである.$x(n)$と$y(n)$を用いて最小2乗法により$(g(0),g(1),g(2),g(3),g(4))$を求めよ.
  \end{itembox}

  以下の式をガウスザイゼル法を用いて解くことで$g(0) \sim g(4)$を求める.

  \begin{eqnarray}
    \left(
    \begin{array}{cccc}
    < x_n x_n > & < x_n x_{n-1} > & \cdots & < x_n x_{n-4} > \\
    < x_{n-1} x_n > & < x_{n-1} x_{n-1} > & \cdots & < x_{n-1} x_{n-4} >\\
    < x_{n-2} x_n > & < x_{n-2} x_{n-1} > & \cdots & < x_{n-2} x_{n-4} > \\
    < x_{n-3} x_n > & < x_{n-3} x_{n-1} > & \cdots & < x_{n-3} x_{n-4} > \\
    < x_{n-4} x_n > & < x_{n-4} x_{n-1} > & \cdots & < x_{n-4} x_{n-4} > \\
    \end{array}
    \right) \left(
      \begin{array}{c}
        g(0) \\
        g(1) \\
        g(2) \\
        g(3) \\
        g(4) \\
      \end{array}
    \right) = \left(
      \begin{array}{c}
        < x_{n} y_{n} > \\
        < x_{n-1} y_{n} > \\
        < x_{n-2} y_{n} > \\
        < x_{n-3} y_{n} > \\
        < x_{n-4} y_{n} > \\
      \end{array}
    \right)  \notag
    \end{eqnarray}

    算出した結果を以下に示す.
    \begin{table}[h]
      \caption{真値と算出した値}
      \label{a}
      \centering
      \begin{tabular}{|c|c|c|} \hline
        & 真値 & 算出した値 \\ \hline
        $g(0)$ & 1.0 & 1.0006358 \\ \hline
        $g(1)$ & -0.9 & -0.89831411 \\ \hline
        $g(2)$ & 0.8 & 0.79984317 \\ \hline
        $g(3)$ & -0.7 & -0.70179211 \\ \hline
        $g(4)$ & 0.6 & 0.60227514 \\ \hline
        \end{tabular}
    \end{table}
    表\ref{a}より,最小2乗法を用いて正しく$g(0) \sim g(4)$を求めることができていることが確認できた.
    今回の用いたプログラムを以下に示す.


    \newpage
    \begin{lstlisting}[language=python]
      import numpy as np

      def gaussSeidel( A, b, tol ):
        xOld = np.empty_like(b)
        error = 1e12

        L = np.tril(A)
        U = A - L
        LInv = np.linalg.inv(L)

        while error > tol:
          x = np.dot( LInv, b-np.dot( U, xOld ) )
          error = np.linalg.norm( x - xOld )/np.linalg.norm(x)
          xOld = x
        return x

      N = 1000
      m = 5
      g = [1.0,-0.9,0.8,-0.7,0.6]
      uniform_rand = np.zeros((2,N))
      gauss_rand = np.zeros((2,N))
      y = np.zeros(N)

      for i in range(N):
        uniform_rand[0][i] = np.random.rand()
        uniform_rand[1][i] = np.random.rand()
        gauss_rand[0][i] = np.sqrt(-2*np.log(uniform_rand[0][i]))*np.cos(2*np.pi*uniform_rand[1][i])
        gauss_rand[1][i] = np.sqrt(-2*np.log(uniform_rand[0][i]))*np.sin(2*np.pi*uniform_rand[1][i])
        for j in range(m):
          y[i] += g[j]*gauss_rand[0][i-j]
        y[i] = y[i] + ((gauss_rand[1][i])/20)

      x = np.zeros(m)
      x = np.append(x,gauss_rand[0])
      cov = np.zeros((m,m))
      xy_cov = np.zeros(m,)

      for i in range(m):
        for j in range(m):
          cov[i][j] = cov[j][i] = np.cov(x[m-i:N+m-i],x[m-j:N+m-j],bias=True)[0,1]
          cov[i][i] = np.cov(x[m-i:N+m-i],x[m-i:N+m-i],bias=True)[0,1]

      for i in range(5):
        xy_cov[i] = np.cov(x[m-i:N+m-i],y[0:1000],bias=True)[0,1]

      x = gaussSeidel(cov,xy_cov,1e-20)

      print(x)
    \end{lstlisting}
\end{document}